% Alternative ablation table with side-by-side component comparison

\begin{table*}[t]
\centering
\caption{Ablation study: Systematic evaluation of FedSat components under severe label skew (Dirichlet $\beta{=}0.3$, 100 clients). Left: component-wise contribution. Right: top-$p$ sensitivity on CIFAR-10.}
\label{tab:ablation-sidebyside}
\begin{minipage}{0.55\textwidth}
\centering
\subcaption{Component Analysis}
\small
\begin{tabular}{@{}lcccc@{}}
\toprule
& \multicolumn{2}{c}{\textbf{Components}} & \multicolumn{2}{c}{\textbf{Accuracy (\%)}} \\
\cmidrule(lr){2-3} \cmidrule(lr){4-5}
\textbf{Dataset} & CACS & FedSat & Value & $\Delta$ \\
\midrule
\multirow{3}{*}{CIFAR-10} 
  & \xmark & \xmark & 60.47 & -- \\
  & \cmark & \xmark & 71.41 & +10.94 \\
  & \cmark & \cmark & \textbf{72.76} & +12.29 \\
\midrule
\multirow{3}{*}{FMNIST} 
  & \xmark & \xmark & 80.61 & -- \\
  & \cmark & \xmark & \textbf{84.39} & +3.78 \\
  & \cmark & \cmark & 83.92 & +3.31 \\
\midrule
\multirow{3}{*}{CIFAR-100} 
  & \xmark & \xmark & 49.69 & -- \\
  & \cmark & \xmark & 50.91 & +1.22 \\
  & \cmark & \cmark & \textbf{51.29} & +1.60 \\
\bottomrule
\end{tabular}
\end{minipage}%
\hfill
\begin{minipage}{0.42\textwidth}
\centering
\subcaption{Top-$p$ Parameter Sensitivity}
\small
\begin{tabular}{@{}lcc@{}}
\toprule
\textbf{Top-$p$} & \textbf{Acc. (\%)} & \textbf{$\Delta$} \\
\midrule
-- (baseline)  & 71.41 & -- \\
\midrule
$p=1$          & 73.38 & +1.97 \\
$p=2$          & \textbf{73.82} & +2.41 \\
$p=4$          & 72.76 & +1.35 \\
$p=5$          & 73.37 & +1.96 \\
$p=10$         & 71.95 & +0.54 \\
\bottomrule
\end{tabular}
\vspace{2mm}

\small
\textit{Baseline: CACS with FedAvg aggregation on CIFAR-10.}
\end{minipage}
\end{table*}

\paragraph{Analysis.}
\textbf{Left table:} CACS (confusion-calibrated local loss) delivers the majority of gains, particularly on CIFAR-10 where inter-class confusion is high (+10.94\%). FedSat's top-$p$ aggregation provides an additional +1.35\% by prioritizing struggling classes at the server. On FMNIST, CACS alone peaks at 84.39\%, suggesting effective local calibration may reduce the need for aggressive server-side selection.

\textbf{Right table:} Optimal $p$ values lie in $\{2, 4, 5\}$ for 10-class CIFAR-10, with peak performance at $p{=}2$ (73.82\%). This suggests \textit{focusing on 1--2 most confused classes} is sufficient. Too small ($p{=}1$) still improves (+1.97\%), while too large ($p{=}10$) dilutes the effect (+0.54\%).
